\documentclass{cv}
\usepackage[dvipsnames]{xcolor}
\usepackage[left=0.75in,top=0.6in,right=0.75in,bottom=0.1in]{geometry}
\newcommand{\tab}[1]{\hspace{.2667\textwidth}\rlap{#1}}
\newcommand{\itab}[1]{\hspace{0em}\rlap{#1}}

\name{张延涛}
\address{意向城市: 北京 \\ 意向岗位: 嵌入式软件开发}
\address{英语六级 (CET-6) \\ 政治面貌: 中共党员}
\address{(+86) 156 3285 5317 \\ \href{mailto:yantao.z@qq.com}{yantao.z@qq.com}}
\address{Github: \href{https://github.com/zhyantao}{zhyantao} \\ Blog: \href{https://zhyantao.github.io/docs}{zhyantao.github.io/docs}}

\renewenvironment{rSection}[1]{
\sectionskip
\textcolor{RoyalPurple}{\MakeUppercase{#1}}
\sectionlineskip
\hrule
\begin{list}{}{
\setlength{\leftmargin}{1.5em}
}
\item[]
}{
\end{list}
}

\begin{document}

%----------------------------------------------------------------------------------------
% EDUCATION SECTION
%----------------------------------------------------------------------------------------
\begin{rSection}{教育背景}
	{\bf 西安交通大学} \hfill {2020.09 - 2023.06} \\
	硕士 | 软件工程 | 软件学院, 电信学部

	{\bf 华北理工大学} \hfill {2016.09 - 2020.06} \\
	学士 | 电子信息工程 | 人工智能学院
\end{rSection}

%----------------------------------------------------------------------------------------
% WORK EXPERIENCE SECTION
%----------------------------------------------------------------------------------------
\begin{rSection}{工作与项目经历}
	\begin{rSubsection}{中兴通讯股份有限公司}{2023.07 - 至今}{嵌入式软件开发工程师}{西安}
		\item \textbf{ZM9300模组定位服务开发}: 负责定位服务模块的设计与实现,基于HD8040D/BG1101芯片对GPSD开源代码进行深度定制与功能扩展,提升定位精度与稳定性。
		\item \textbf{GNSS模块维护与SDK开发}: 负责GNSS模块的代码维护、功能迭代与性能优化;负责开发配套SDK及自动化测试用例,累计处理并修复客户反馈及内部BUG 200余项,提升模块可靠性。
		\item \textbf{高精度时间同步服务}: 负责时间同步服务的开发与维护,适配chronyd源码并基于自研芯片实现gPTP协议栈;外场测试表明,时间偏移(offset)等关键指标与高通方案相当。
		\item \textbf{启动流程优化与可视化}: 基于自研芯片适配bootchard,首次在S1平台实现开机流程可视化,为系统启动性能分析与优化提供了关键数据支撑与调试手段。
	\end{rSubsection}

	\begin{rSubsection}{中国重型机械研究院股份公司}{2021.09 - 2022.05}{后端开发工程师(校企合作)}{西安}
		\item \textbf{连铸工艺仿真平台}: 设计并实现基于Spring Boot的连铸智能工艺仿真软件包,对关键工艺流程进行高保真模拟,构建了可配置、可视化的工艺优化与决策支持平台。
		\item \textbf{后端架构与核心功能开发}: 负责系统后端架构设计,采用Spring Boot + MySQL技术栈,通过JPA完成业务管理模块的CRUD操作与数据持久化,保障了系统稳定与数据一致性。
		\item \textbf{工艺参数计算与集成}: 集成JMatPro工具完成核心工艺参数的自动化计算与分析,并对结果数据进行结构化处理与封装,为前端可视化模块提供准确、高效的数据接口。
		\item \textbf{前端界面开发协作}: 参与前端部分开发,使用Vue.js实现多个界面的数据展示与交互逻辑,确保了前后端数据联动的准确性与用户体验的流畅性。
	\end{rSubsection}
\end{rSection}

%----------------------------------------------------------------------------------------
% TECHNICAL STRENGTHS SECTION
%----------------------------------------------------------------------------------------
\begin{rSection}{专业技能}
	% 移除了空白的{rSubsection},直接使用item列表,更简洁。
	\begin{rSubsection}{}{}{}{}
		\item \textbf{编程语言}: 熟悉 C/C++, 具备良好的面向对象设计与编程能力,代码规范;掌握 Java,可使用 Spring Boot 进行基本的后端开发。
		\item \textbf{开发工具}: 熟练使用 Git、Shell、GCC、GDB、Makefile、Yocto 等开发与构建工具。
		\item \textbf{计算机基础}: 熟悉常见数据结构与算法、操作系统原理,具备良好的系统设计能力和业务理解能力。
	\end{rSubsection}
\end{rSection}

%----------------------------------------------------------------------------------------
% AWARDS SECTION
%----------------------------------------------------------------------------------------
\begin{rSection}{获奖情况} \itemsep -2pt
	{ASC 超级计算机竞赛}\hfill {\em 国家级二等奖}\hfill {2019 年 4 月} \\
	{大学生数学建模竞赛}\hfill  {\em { }省部级二等奖}\hfill {2017 年 9 月}
\end{rSection}

\end{document}
