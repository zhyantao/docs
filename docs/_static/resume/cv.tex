%%%%%%%%%%%%%%%%%%%%%%%%%%%%%%%%%%%%%%%%%
% Medium Length Professional CV
% LaTeX Template
% Version 2.0 (8/5/13)
%
% This template has been downloaded from:
% http://www.LaTeXTemplates.com
%
% Original author:
% Trey Hunner (http://www.treyhunner.com/)
%
% Important note:
% This template requires the cv.cls file to be in the same directory as the
% .tex file. The cv.cls file provides the resume style used for structuring the
% document.
%
%%%%%%%%%%%%%%%%%%%%%%%%%%%%%%%%%%%%%%%%%

%----------------------------------------------------------------------------------------
%	PACKAGES AND OTHER DOCUMENT CONFIGURATIONS
%----------------------------------------------------------------------------------------

\documentclass{cv} % Use the custom cv.cls style
\usepackage[dvipsnames]{xcolor}

\usepackage[left=0.75in,top=0.6in,right=0.75in,bottom=0.1in]{geometry} % Document margins
\newcommand{\tab}[1]{\hspace{.2667\textwidth}\rlap{#1}}
\newcommand{\itab}[1]{\hspace{0em}\rlap{#1}}
\name{张延涛} % Your name
\address{(+86)156-3285-5317 \\ \href{mailto:yantao.z@qq.com}{yantao.z@qq.com}}
\address{Github: \href{https://github.com/zhyantao}{zhyantao} \\ Blog: \href{https://zhyantao.github.io/docs}{zhyantao.github.io/docs}}
\address{籍贯: 河北省\ 衡水市 \\ 常住地: 陕西省\ 西安市}

\renewenvironment{rSection}[1]{
\sectionskip
\textcolor{RoyalPurple}{\MakeUppercase{#1}}
\sectionlineskip
\hrule
\begin{list}{}{
\setlength{\leftmargin}{1.5em}
}
\item[]
}{
\end{list}
}



\begin{document}

%----------------------------------------------------------------------------------------
%	EDUCATION SECTION
%----------------------------------------------------------------------------------------
\begin{rSection}{教育背景}

  {\bf \href{https://www.xjtu.edu.cn}{西安交通大学}} (硕士) \hfill {\em 2020 - 2023}
  \smallskip \\ 软件工程, 软件学院, 电信学部 \hfill

  {\bf \href{https://www.ncst.edu.cn}{华北理工大学}} (本科) \hfill {\em 2016 - 2020}
  \smallskip \\ 电子信息工程, 人工智能学院 \hfill

\end{rSection}

%----------------------------------------------------------------------------------------
%	WORK EXPERIENCE SECTION
%----------------------------------------------------------------------------------------
\begin{rSection}{项目经历}

  \begin{rSubsection}{中兴通讯股份有限公司}{2023年7月 - 至今}{嵌入式软件开发}{陕西西安}
    \item 主导ZM9300国产化项目中的定位与授时业务, 基于中兴自研S1芯片深度定制gPTP协议, 弥补国产芯片设计短板, 外场测试offset指标优于高通和华为平台.
    \item 承接GNSS业务的SDK的开发工作, 基于HD8040D和BG1101芯片维护已有功能及业务扩展, 适配gpsd实现RTK和DR功能的开发和验证, 累计修复故障161项.
    \item 攻克比亚迪授时难题, 通过修改chrony实现, 优化中兴模组时间同步慢的问题,将时间同步从1分钟优化至5s内, 速度提升92\%, 显著提升了系统的实时性.
    \item 深度定制bootchard功能, 首次在S1平台实现开机流程可视化, 为后续系统启动性能优化提供了数据基础和技术手段.
  \end{rSubsection}

  %------------------------------------------------

  \begin{rSubsection}{中国重型机械研究院股份公司}{2021年9月 - 2022年5月}{后端开发工程师}{陕西西安}
    \item 设计并实现基于Spring Boot的连铸智能工艺仿真软件包, 有效模拟连铸生产过程中的关键工艺流程, 为工艺优化与决策支持提供了可视化、可配置的仿真平台.
    \item 负责系统后端架构设计与功能开发, 采用Spring Boot框架结合MySQL数据库, 完成了业务流程管理模块的增删改查功能, 并通过JPA实现数据持久化操作.
    \item 集成JMatPro工具完成核心工艺参数的计算与分析, 对计算结果进行整理与封装, 为前端数据可视化模块提供结构化数据支撑.
    \item 参与前端页面开发与维护, 使用Vue.js实现部分前端界面的展示与交互功能, 保障前后端数据联动的准确性与响应效率.
  \end{rSubsection}

\end{rSection}

%----------------------------------------------------------------------------------------
%	TECHNICAL STRENGTHS SECTION
%----------------------------------------------------------------------------------------
\begin{rSection}{专业技能}
  \begin{rSubsection}{}{}{}{}
    \item 掌握操作系统基础知识, 熟悉进程调度、内存管理、文件系统等核心概念.
    \item 熟练掌握 C/C++, 具备良好的面向对象设计与编程能力, 代码风格规范, 注释清晰.
    \item 熟悉 Java 开发, 熟练使用 Spring Boot 构建 Web 应用, 掌握 JPA 进行数据持久化.
    \item 了解前端开发技术, 熟悉 HTML、CSS 与 JavaScript, 可基于 Vue.js 开发单页面应用.
    \item 熟悉 Git、Shell 脚本、GCC 编译器、 Makefile 及 Yocto 等常用开发与构建工具.
  \end{rSubsection}
\end{rSection}

%	EXAMPLE SECTION
%----------------------------------------------------------------------------------------

\begin{rSection}{获奖情况} \itemsep -2pt
  {大学生数学建模竞赛}\hfill {\it 省部级二等奖}\hfill {\em 2017年9月} \\
  {ASC超级计算机竞赛}\hfill {\it 国家级二等奖}\hfill {\em 2019年4月}
\end{rSection}

\end{document}
