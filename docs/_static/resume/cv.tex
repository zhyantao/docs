%%%%%%%%%%%%%%%%%%%%%%%%%%%%%%%%%%%%%%%%%
% Medium Length Professional CV
% LaTeX Template
% Version 2.0 (8/5/13)
%
% This template has been downloaded from:
% http://www.LaTeXTemplates.com
%
% Original author:
% Trey Hunner (http://www.treyhunner.com/)
%
% Important note:
% This template requires the cv.cls file to be in the same directory as the
% .tex file. The cv.cls file provides the resume style used for structuring the
% document.
%
%%%%%%%%%%%%%%%%%%%%%%%%%%%%%%%%%%%%%%%%%

%----------------------------------------------------------------------------------------
% PACKAGES AND OTHER DOCUMENT CONFIGURATIONS
%----------------------------------------------------------------------------------------

\documentclass{cv} % Use the custom cv.cls style
\usepackage[dvipsnames]{xcolor}

\usepackage[left=0.75in,top=0.6in,right=0.75in,bottom=0.1in]{geometry} % Document margins
\newcommand{\tab}[1]{\hspace{.2667\textwidth}\rlap{#1}}
\newcommand{\itab}[1]{\hspace{0em}\rlap{#1}}
\name{张延涛}
\address{(+86)156-3285-5317 \\ \href{mailto:yantao.z@qq.com}{yantao.z@qq.com}}
\address{Github: \href{https://github.com/zhyantao}{zhyantao} \\ Blog: \href{https://zhyantao.github.io/docs}{zhyantao.github.io/docs}}
\address{现居地: 陕西省\ 西安市 \\ 意向城市: 北京\ 西安\ 苏州}

\renewenvironment{rSection}[1]{
\sectionskip
\textcolor{RoyalPurple}{\MakeUppercase{#1}}
\sectionlineskip
\hrule
\begin{list}{}{
\setlength{\leftmargin}{1.5em}
}
\item[]
}{
\end{list}
}

\begin{document}
%----------------------------------------------------------------------------------------
% EDUCATION SECTION
%----------------------------------------------------------------------------------------
\begin{rSection}{教育背景}
	{\bf 西安交通大学} 硕士 \hfill {2020 年 9 月 - 2023 年 6 月}
	\smallskip \\ 软件工程, 软件学院, 电信学部

	{\bf 华北理工大学} 本科 \hfill {2016 年 9 月 - 2020 年 6 月}
	\smallskip \\ 电子信息工程, 人工智能学院 \hfill
\end{rSection}

%----------------------------------------------------------------------------------------
% WORK EXPERIENCE SECTION
%----------------------------------------------------------------------------------------
\begin{rSection}{项目经历}
	\begin{rSubsection}{中兴通讯股份有限公司}{2023 年 7 月 - 至今}{嵌入式软件开发}{陕西西安}
		\item 主导 ZM9300 国产化项目中的定位与授时业务, 基于中兴自研 S1 芯片深度定制 gPTP 协议, 弥补国产芯片设计短板, 外场测试 offset 指标优于高通和华为平台.
		\item 承接 GNSS 业务的 SDK 的开发工作, 基于 HD8040D 和 BG1101 芯片维护已有功能及业务扩展, 适配 gpsd 实现 RTK 和 DR 功能的开发和验证, 累计修复故障 161 项.
		\item 攻克比亚迪授时难题, 通过修改 chrony 实现, 优化中兴模组时间同步慢的问题,将时间同步从 1 分钟优化至 5s 内, 速度提升 92\%, 显著提升了系统的实时性.
		\item 深度定制 bootchard 功能, 首次在 S1 平台实现开机流程可视化, 为后续系统启动性能优化提供了数据基础和技术手段.
	\end{rSubsection}
	%------------------------------------------------
	\begin{rSubsection}{中国重型机械研究院股份公司}{2021 年 9 月 - 2022 年 5 月}{后端开发工程师}{陕西西安}
		\item 设计并实现基于 Spring Boot 的连铸智能工艺仿真软件包, 有效模拟连铸生产过程中的关键工艺流程, 为工艺优化与决策支持提供了可视化、可配置的仿真平台.
		\item 负责系统后端架构设计与功能开发, 采用 Spring Boot 框架结合 MySQL 数据库, 完成了业务流程管理模块的增删改查功能, 并通过 JPA 实现数据持久化操作.
		\item 集成 JMatPro 工具完成核心工艺参数的计算与分析, 对计算结果进行整理与封装, 为前端数据可视化模块提供结构化数据支撑.
		\item 参与前端页面开发与维护, 使用 Vue.js 实现部分前端界面的展示与交互功能, 保障前后端数据联动的准确性与响应效率.
	\end{rSubsection}
\end{rSection}

%----------------------------------------------------------------------------------------
% TECHNICAL STRENGTHS SECTION
%----------------------------------------------------------------------------------------
\begin{rSection}{专业技能}
	\begin{rSubsection}{}{}{}{}
		\item 掌握操作系统基础知识, 熟悉进程调度、内存管理、文件系统等核心概念.
		\item 熟练掌握 C/C++, 具备良好的面向对象设计与编程能力, 代码风格规范, 注释清晰.
		\item 熟悉 Java 开发, 熟练使用 Spring Boot 构建 Web 应用, 掌握 JPA 进行数据持久化.
		\item 了解前端开发技术, 熟悉 HTML、CSS 与 JavaScript, 可基于 Vue.js 开发单页面应用.
		\item 熟悉 Git、Shell 脚本、GCC 编译器、 Makefile 及 Yocto 等常用开发与构建工具.
	\end{rSubsection}
\end{rSection}

% EXAMPLE SECTION
%----------------------------------------------------------------------------------------
\begin{rSection}{获奖情况} \itemsep -2pt
	{大学生数学建模竞赛}\hfill {\em 省部级二等奖}\hfill {2017 年 9 月} \\
	{ASC 超级计算机竞赛}\hfill {\em 国家级二等奖}\hfill {2019 年 4 月}
\end{rSection}

\end{document}
