\documentclass[a4paper]{ctexart}
\usepackage[landscape]{geometry}
\usepackage{multicol}
\usepackage{tikz}
\usepackage{listings}

\makeatletter

\advance\topmargin-.8in
\advance\textheight3in
\advance\textwidth3in
\advance\oddsidemargin-1.5in
\advance\evensidemargin-1.5in
\parindent0pt
\parskip2pt

\begin{document}

\begin{multicols*}{3}

  \tikzstyle{mybox} = [draw=black, fill=white, very thick,
  rectangle, rounded corners, inner sep=10pt, inner ysep=10pt]
  \tikzstyle{fancytitle} =[fill={rgb:red,220;green,220;blue,221},
  text=white, font=\bfseries]
  %------------ START OF CONTENT -------------


  %------------- START SECTION ---------------
  \begin{tikzpicture}
    \node [mybox] (box){%
      \begin{minipage}{0.3\textwidth}

        \begin{verbatim}
1. 下标 []、函数调用 ()、结构选择 .
2. 一元运算符 指针 *、自增 ++、自减 --
3. 二元运算符
   数学运算符 *、/、%、+、-
   移位运算符 <<、>> \\
   关系运算符 <、>、<=、>=
   逻辑运算符 &、^、|
   赋值运算符 =、*=、+=、&=、<<=
   条件运算符(三元运算符) ? :
4. 逗号运算符 ,
\end{verbatim}

      \end{minipage}
    };
    %-----------------------------------------
    \node[fancytitle, right=10pt] at (box.north west) {运算符的优先级};
  \end{tikzpicture}



  %------------- START SECTION ---------------
  \begin{tikzpicture}
    \node [mybox] (box){%
      \begin{minipage}{0.3\textwidth}

        \verb!NULL! 既表示 0 也表示空指针,存在二义性。\\
        \verb!'\0'! 常作为字符串的结尾。\\
        \verb!nullptr! 是为了消除 \verb!NULL! 的二义性而创造的。

      \end{minipage}
    };
    %-----------------------------------------
    \node[fancytitle, right=10pt] at (box.north west) {\verb!NULL!、\verb!'\0'!、\verb!0! 之间的区别};
  \end{tikzpicture}

  %------------- START SECTION ---------------
  \begin{tikzpicture}
    \node [mybox] (box){%
      \begin{minipage}{0.3\textwidth}

        \begin{verbatim}
// 初始化一维数组
int arr[10]{ 0 };
int arr[] = {0, 1, 2};
// 初始化二维数组
int arr[3][4] = { 0 };
int arr[3][3] = {{0,1}, {0,0,2}, {3}};
// 初始化单个字符串
char *str = "ABCD";
char str[] = "ABCD";
char str[] = {'A', 0};
// 初始化多个字符串
char *str[] = {"ABC", "DEF"};
\end{verbatim}

      \end{minipage}
    };
    %-----------------------------------------
    \node[fancytitle, right=10pt] at (box.north west) {初始化数组或字符串};
  \end{tikzpicture}


  %------------- START SECTION ---------------
  \begin{tikzpicture}
    \node [mybox] (box){%
      \begin{minipage}{0.3\textwidth}

        \begin{verbatim}
char arr[] = "hello";
arr[0] = 'X'; // 正确,可以修改栈区
char *ptr = "world";
ptr[0] = 'X'; // 错误,不能修改常量区
\end{verbatim}

      \end{minipage}
    };
    %-----------------------------------------
    \node[fancytitle, right=10pt] at (box.north west) {修改数组或字符串};
  \end{tikzpicture}


  %------------- START SECTION ---------------
  \begin{tikzpicture}
    \node [mybox] (box){%
      \begin{minipage}{0.3\textwidth}

        \begin{verbatim}
// 复制到栈区
char str[] = "hello";
char buf[10];              // 开辟栈区空间
int sz = sizeof(str) / sizeof(str[0]);
//strcpy(buf, str);        // UNIX 习惯
strcpy_s(buf, sz, str);    // 不能用 b = a
strcmp(buf, str) == 0;     // 不能用 b == a
// 复制到堆区
size_t len = strlen(str);
char *p = NULL;         // NULL 存在二义性
p = (char *)malloc(sizeof(char) * (len + 1));
strcpy_s(buf, len + 1, str);// 不能忘记 + 1
if (strcmp(buf, str) == 0)  // 不能用 b == a
  free(p);              // 释放动态开辟的空间
p = NULL;               // 避免出现野指针
\end{verbatim}

      \end{minipage}
    };
    %-----------------------------------------
    \node[fancytitle, right=10pt] at (box.north west) {复制字符串};
  \end{tikzpicture}


  %------------- START SECTION ---------------
  \begin{tikzpicture}
    \node [mybox] (box){%
      \begin{minipage}{0.3\textwidth}

        \begin{verbatim}
// 申请和释放一维数组
int *num = (int *)malloc(sizeof(int) * 1024);
if (num != NULL)
  free(num);
// 申请和释放二维数组
char **ptr = (char**)malloc(sizeof(char*) * 1024);
for (int i = 0; i < 1024; i++)
  ptr[i] = (char *)malloc(sizeof(char) * 30);
for (int i = 0; i < 1024; i++)
  if (ptr[i] != NULL)
    free(ptr[i]);
if (ptr != NULL)
  free(ptr);
\end{verbatim}

      \end{minipage}
    };
    %-----------------------------------------
    \node[fancytitle, right=10pt] at (box.north west) {动态开辟空间 \verb!malloc / free!};
  \end{tikzpicture}


  %------------- START SECTION ---------------
  \begin{tikzpicture}
    \node [mybox] (box){%
      \begin{minipage}{0.3\textwidth}

        \begin{verbatim}
ptr = (char *) str.c_str();// string -> char*
double d = atof("0.23");  // string -> double
int i = atoi("1021");     // string -> int
long l = atol("303992");  // string -> long
sprintf(chs, "%f", 2.3);  // 保存到 chs
char *ptr = chs;          // char[] -> char*
strcpy(chs, ptr, len);    // char* -> char[]
string str = chs;         // char[] -> string
(unsigned)num;            // 强制转换
\end{verbatim}

      \end{minipage}
    };
    %-----------------------------------------
    \node[fancytitle, right=10pt] at (box.north west) {类型转换 \verb!char chs[100]{ 0 }!};
  \end{tikzpicture}


  %------------- START SECTION ---------------
  \begin{tikzpicture}
    \node [mybox] (box){%
      \begin{minipage}{0.3\textwidth}

        \begin{verbatim}
// 创建哈希结构
struct my_struct {
  int id;             // key
  char name[10];      // value
  UT_hash_handle hh;  // make it hashable
};
// 定义哈希表
struct my_struct *users = NULL;
// 向哈希表中添加数据
void add_user(struct my_struct *s) {
  HASH_ADD_INT( users, id, s );
}
// 从哈希表中检索数据
struct my_struct *find_user(int user_id) {
  struct my_struct *s;

  HASH_FIND_INT( users, &user_id, s );
  return s;
}
// 删除哈希表中的一条数据
void delete_user(struct my_struct *user) {
  HASH_DEL( users, user);
}
\end{verbatim}

      \end{minipage}
    };
    %-----------------------------------------
    \node[fancytitle, right=10pt] at (box.north west) {哈希表 \verb!#include "uthash.h"!};
  \end{tikzpicture}


  %------------ END OF CONTENT ---------------
\end{multicols*}

\end{document}
