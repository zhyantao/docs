%%%%%%%%%%%%%%%%%%%%%%%%%%%%%%%%%%%%%%%%%
% Medium Length Professional CV
% LaTeX Template
% Version 2.0 (8/5/13)
%
% This template has been downloaded from:
% http://www.LaTeXTemplates.com
%
% Original author:
% Trey Hunner (http://www.treyhunner.com/)
%
% Important note:
% This template requires the cv.cls file to be in the same directory as the
% .tex file. The cv.cls file provides the resume style used for structuring the
% document.
%
%%%%%%%%%%%%%%%%%%%%%%%%%%%%%%%%%%%%%%%%%

%----------------------------------------------------------------------------------------
%	PACKAGES AND OTHER DOCUMENT CONFIGURATIONS
%----------------------------------------------------------------------------------------

\documentclass{cv} % Use the custom cv.cls style
\usepackage[dvipsnames]{xcolor}

\usepackage[left=0.75in,top=0.6in,right=0.75in,bottom=0.1in]{geometry} % Document margins
\newcommand{\tab}[1]{\hspace{.2667\textwidth}\rlap{#1}}
\newcommand{\itab}[1]{\hspace{0em}\rlap{#1}}
\name{Yantao Zhang} % Your name
\address{Xi'an, 710049 \\ Shaanxi, China} % Your address
%\address{123 Pleasant Lane \\ City, State 12345} % Your secondary addess (optional)
\address{Github: \href{https://github.com/zhyantao}{zhyantao} \\ Blog: \href{https://getstarted.readthedocs.io}{getstarted.rtfd.io}}
\address{(+86)156-3285-5317 \\ \href{mailto:yantao.z@outlook.com}{yantao.z@outlook.com}} % Your phone number and email


\renewenvironment{rSection}[1]{
\sectionskip
\textcolor{RoyalPurple}{\MakeUppercase{#1}}
\sectionlineskip
\hrule
\begin{list}{}{
\setlength{\leftmargin}{1.5em}
}
\item[]
}{
\end{list}
}



\begin{document}

%----------------------------------------------------------------------------------------
%	EDUCATION SECTION
%----------------------------------------------------------------------------------------

\begin{rSection}{Education}


  {\bf \href{https://www.xjtu.edu.cn}{Xi'an Jiaotong University}} \hfill {\em 2020 - 2023}
  \\ MA in Software Engineering, 2023 \hfill
  \\ School of Software Engineering, 2020 \hfill

  {\bf \href{https://www.ncst.edu.cn}{North China University of Science and Technology}} \hfill {\em 2016 - 2020}
  \\ BA in Electronic Information Engineering, 2020 \hfill
  \\ College of Artificial Intelligence, 2016 \hfill

  %Minor in Linguistics \smallskip \\
  %Member of Eta Kappa Nu \\
  %Member of Upsilon Pi Epsilon \\


\end{rSection}
%----------------------------------------------------------------------------------------
%	TECHNICAL STRENGTHS SECTION
%----------------------------------------------------------------------------------------

\begin{rSection}{Coding Skills }

  \begin{tabular}{ @{} >{\bfseries}l @{\hspace{6ex}} l }
    Programming Languages & C/C++, Python, Java, MySQL                          \\
    C++ Libaries          & pthread\_*, mutex, condition\_variable, socket, shm \\
    Python Packages       & PyTorch, Matplotlib, NumPy, Pandas                  \\
    Java Packages         & SpringBoot, JNI, JPA                                \\
    Software \& Tools     & Git, Shell, GCC, Makefile                           \\
  \end{tabular}

\end{rSection}

%----------------------------------------------------------------------------------------
%	WORK EXPERIENCE SECTION
%----------------------------------------------------------------------------------------

\begin{rSection}{Experience}

  \begin{rSubsection}{Zhongxing Telecommunication Equipment (ZTE) Corporation}{July 2023 - Now}{Embedded Software Engineer}{}
    \item Maintain GPSD source code and make some changes to meet the requirements.
    \item Wrote an API to transfer raw data to upper-layer application.
    \item Fix some known bugs, such as GPSD cannot catch all binary data from serial port.
  \end{rSubsection}

  %------------------------------------------------

  \begin{rSubsection}{Code Analyzer - Github (Ongoing)}{April 2024 - Now}{LLM Application Developer}{}
    \item Using Clang and Llama2 to analyze C/C++ code.
  \end{rSubsection}

  %------------------------------------------------

  \begin{rSubsection}{MIT 6.828 Operating System - Github (Ongoing)}{April 2024 - Now}{Student}{}
    \item Pass all the exam.
  \end{rSubsection}

\end{rSection}


%	EXAMPLE SECTION
%----------------------------------------------------------------------------------------

\begin{rSection}{Achievements} \itemsep -2pt
  {Kaggle Competition}\hfill {\em 2024} \\
  {Codeforces Orange}\hfill {\em 2024}
\end{rSection}




\end{document}
